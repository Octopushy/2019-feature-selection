%%%% using 'arara' 4.0
% arara: xelatex: {synctex: yes, interaction: nonstopmode}
% arara: bibtex
% arara: xelatex: {synctex: yes, interaction: nonstopmode}
% arara: xelatex: {synctex: yes, interaction: nonstopmode}

% arara: indent: {overwrite: yes}

% arara: clean: { extensions: [aux, bcf, cod, blg, lof, lot, out, toc, log, xml, bak0 ] }

\documentclass[review]{elsarticle}

% Figures Links, mittig und rechts platzieren
\usepackage[export]{adjustbox}
\usepackage{caption}
\usepackage{subcaption}
\usepackage{amsmath}

% prevents that appendices are moved behind references
\usepackage{placeins}

\usepackage[nolist]{acronym}

\usepackage{longtable}
\usepackage{booktabs}
\usepackage{multirow}
\usepackage{float}

% enable linking to subsubsection
\setcounter{secnumdepth}{3}

% various symbols, e.g. \degree
\usepackage{gensymb}

\usepackage[hidelinks]{hyperref}

\usepackage{lineno}
\modulolinenumbers[5]

% set autoref abbr for appendix
\newcommand*{\Appendixautorefname}{appendix}

\journal{Journal "Remote Sensing of environment"}

% line breaks in table cells
\newcommand{\specialcell}[2][l]{%
  \begin{tabular}[#1]{@{}l@{}}#2\end{tabular}}

% tilde
\newcommand{\mytilde}{\raise.17ex\hbox{$\scriptstyle\mathtt{\sim}$}}

%% APA style
\bibliographystyle{model5-names}\biboptions{authoryear}

\begin{document}

\begin{frontmatter}

	\title{title}

	%% Group authors per affiliation:
	\author[FSU]{Patrick Schratz}
	\cortext[mycorrespondingauthor]{Corresponding author}
	\ead{patrick.schratz@uni-jena.de}

	\author[FSU]{Jannes Muenchow}
	\author[NEIKER]{Eugenia Iturritxa}
	%\author[TUDO]{Jakob Richter}
	\author[FSU]{Alexander Brenning}

	\address[FSU]{Department of Geography, GIScience group, Grietgasse 6, 07743, Jena, Germany}
	%\address[NEIKER]{NEIKER, Granja Modelo –Arkaute, Apdo. 46, 01080 Vitoria-Gasteiz, Arab, Spain}
	%\address[TUDO]{Department of Statistics, TU Dortmund University, Germany}

	\begin{abstract}

	\end{abstract}

	\begin{keyword}
		hyperspectral imagery \sep statistical learning \sep spatial cross-validation
	\end{keyword}

\end{frontmatter}

\linenumbers

% längste Abkürzung steht hier!!! in eckigen Klammern
\begin{acronym}[AUROC]

	% geringerer Zeilenabstand
	%\setlength{\itemsep}{-\parsep}
	\acro{ANN}{Artificial Neural Network}
	\acro{AUROC}{Area Under the Receiver Operating Characteristics Curve}
	\acro{BRT}{Boosted Regression Trees}
	\acro{CART}{Classification and Regression Trees}
	\acro{CV}{cross-validation}
	\acro{ENM}{Environmental Niche Modeling}
	\acro{FPR}{False Positive Rate}
	\acro{GAM}{Generalized Additive Model}
	\acro{GBM}{Gradient Boosting Machine}
	\acro{GLM}{Generalized Linear Model}
	\acro{ICGC}{Institut Cartografic i Geologic de Catalunya}
	\acro{IQR}{Interquartile Range}
	\acro{WKNN}{Weighted $k$-nearest neighbor}
	\acro{MARS}{Multivariate Adaptive Regression Splines}
	\acro{MEM}{Maximum Entropy Model}
	\acro{NBI}{Narrow Band Index}
	\acro{LOWESS}{Locally Weighted Scatter Plot Smoothing}
	\acro{PISR}{Potential Incoming Solar Radiation}
	\acro{RBF}{Radial Basis Function}
	\acro{RF}{Random Forest}
	\acro{SDM}{Species Distribution Modeling}
	\acro{SVM}{Support Vector Machines}
	\acro{TPR}{True Positive Rate}
\end{acronym}

\section{Introduction}
\label{sec:intro}

\section{Data and study area}



\cite{Brenning2012}

% \begin{figure} [t!]
% 	\begin{center}
% 		\makebox[\textwidth]{\includegraphics[width=\textwidth] {../../04_figures/01_data/study_area.pdf}}
% 		\caption[Study area]{Spatial distribution of tree observations within the Basque Country, northern Spain, showing infection state by \textit{Diplodia sapinea}.}
% 		\label{fig: study_area}
% 	\end{center}
% \end{figure}

\subsection{Hyperspectral data}

The airborne hyperspectral data was acquired on September 28th and October 5th 2016 around 12 am.
The images were taken by an AISAEAGLE-II sensor from the \ac{ICGC}.
All preprocessing steps (geometric, radiometric, atmospheric) have been conducted by \ac{ICGC}.

Additional information is provided in Table 1:

% parameter limits
\begin{table}[b!]
\centering
\caption[t]{Specifications of hyperspectral data.}
\begingroup\footnotesize
\begin{tabular}{ll}
	\\
	Characteristic         & Value                               \\
	\hline
	Geometric resolution   & 1 m                                 \\
	Radiometric resolution & 12 bit                              \\
	Spectral resolution    & 126 bands (404.08 nm - 996.31 nm)   \\
	Correction:            & Radiometric, geometric, atmospheric
\end{tabular}
\endgroup
\label{tab:hyperparameter_limits}
\end{table}



\section{Methods}

\subsection{Index calculation}

% link to PDF with veg indeces
All vegetation indices (90 total) suitable for the wavelength range of the hyperspectral data that are offered by the \texttt{hsdar} package have been calculated.
Additionally, all possible \ac{NBI} were calculated from the data using the following formula:

\begin{equation}
	NBI_{i,j} = \frac{B_{i} - B_{j}}{B_{i} + B_{j}}
\end{equation}

\noindent
where $i$ and $j$ are the respective band numbers.

To account for geometric offsets, we calculated every index five times using a buffer from 1 - 5 meter around the centroid of the respective tree.
The mean value of all pixels touched by the buffer was assigned as the final value for each index.
The exact number of contributing pixels of an index cannot be determined as it depends on the location of the tree within the pixel grid.
If a tree is located at the border of a pixel, the same buffer (e.g. 3 m) will include more pixels than if the point is located at the center of a pixel.
Also, if a tree is located at the border of the image data, some directions of the buffer may not contain values.
Missing values were removed from the mean value calculation.
However, the bigger the buffer size, the more pixels are represented by the index value.
In total, 7875 possible \ac{NBI}s have been calculated for each buffer size summing up to 39375 indices.
The same applies to the vegetation indices which sum up to 450 indices leading to a total number of 39825 indices per tree.



\subsection{Ridge regression}



\section*{References}

\bibliography{Biblio_hyperspectral}

\end{document}
